\documentclass[a4paper,12pt]{article}

% Paquetes básicos
\usepackage[utf8]{inputenc}
\usepackage[spanish]{babel}
\usepackage{amsmath, amssymb}
\usepackage{graphicx}
\usepackage{geometry}
\usepackage{hyperref}
\geometry{margin=2.5cm}

% Datos del informe
\title{Inferencia Y Estimación}
\title{Título del Informe}
\author{Catalina Hirsch - 36557 \\ Clara Zavaroni - legajo \\ Maylen Antonella Villagran Cardozo - legajo \\ \\ Universidad de San Andrés}
\date{\today}

\begin{document}

\maketitle

\begin{abstract}
La realización de este trabajo práctico tiene como objetivo aplicar el método de 
PCA en la compresión de imágenes. De esta manera, se busca reducir su espacio de almacenamiento,
minimizando la pérdida de información, conservando los datos que contengan la mayor parte de la información.
Luego, se descomprime la imagen y se compara con la original para evaluar la calidad de la compresión. 
Para ello, se evalúa su desempeño basado en la cantidad de componentes principales utilizados.
Finalmente, pudimos observar que,....
\end{abstract}

\section{Ejercicio 1}
El propósito de este ejercicio consiste en analizar el comportamiento de los píxeles vecinos en las imágenes utilizadas.
En primer lugar, se cargan las imágenes y se convierten a escala de grises. Luego, se divide cada imagen en bloques de 2x1 píxeles
contiguos verticalmente. A continuación, se calcula la correlación entre los píxeles de cada bloque y se almacena en un vector.
Finalmente, se realiza un gráfico de dispersión de las correlaciones obtenidas para cada imagen.
Los resultados se presentan a continuación:


\section{Metodología}
Explique los métodos, materiales y procedimientos utilizados.

\section{Resultados}
Presente los resultados obtenidos, incluyendo tablas y figuras si es necesario.

\section{Discusión}
Analice e interprete los resultados, comparando con la literatura o expectativas.

\section{Conclusiones}
Resuma los hallazgos principales y posibles trabajos futuros.

\section*{Referencias}
\bibliographystyle{plain}
\bibliography{referencias}

\end{document}